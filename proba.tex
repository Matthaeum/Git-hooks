\documentclass{beamer}
\usepackage[croatian]{babel}
\usepackage[utf8]{inputenc}
\usepackage{graphicx}
\usepackage{hyperref}
\usepackage{natbib}
\usepackage{bibentry}

\usetheme{PaloAlto}
\usecolortheme{fly}

\title{\textbf{Git Hooks}}
\subtitle{\textit{autori:} Matej Anić\newline
Karlo Veršić\newline
Ani Perušić
}


\begin{document}
	\frame {
		\titlepage
	}
	\frame {
		\frametitle{Što su Git hooks?}

		\textbf{Git Hooks} su skripte koje Git izvršava prije ili nakon događaja kao što su: \textit{commit, push} i \textit{receive}.
		\newline
		\newline
		Skripte se personaliziraju i prilagođavaju prema potrebi korisnika (npr. pri optimizaciji razvojnog toka, mijenjanju okruženja, te poticanju korištenja 		  			\textit{commit} funkcije).
		
	}
	\frame{
		\frametitle{\textit{Primjeri Git Hooks skripti:}}
		
		\underline{pre-commit}: (provjera poruke uz commit)\newline
		\underline{post-commit}: (slanje obavijesti o novome commitu)
		
	}
	\frame{
		\frametitle{Kako funkcioniraju?}
		Svaki Git repozitorij ima \textit{.git/hooks} direktorij koji sadrži standardne Git Hooks skripte.\newline
		Slobodno se mogu izmjenjivati ili ažurirati, a Git će ih automatski izvršiti kada budu potrebne.
	}
	\frame{
		
	}
	\frame{
		\frametitle{Literatura}
	}
\end{document}